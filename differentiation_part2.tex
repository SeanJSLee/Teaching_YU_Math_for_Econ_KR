\documentclass[aspectratio=169]{beamer}
\usepackage{lectureppt} % Load your custom style
% \usepackage[utf8]{inputenc}



\title{수리경제학}
\subtitle{미분 1: 도함수와 미분}
% \subtitle{직관적 이해와 활용 테스트트}
\author{이재석}
\date{2025-04-15}


\begin{document}

\begin{frame}
  \titlepage
\end{frame}

\begin{frame}{목차}
  \tableofcontents
\end{frame}

% \section{미분을} % 정의하는 방법 1 대수학 (Algebra)적 접근}
% \begin{frame}%{미분을 정의하는 방법 1: 대수학 (Algebra)적 접근}
%   \begin{itemize}
%     \item 함수 \( f(x) \)의 미분계수는 다음과 같이 정의됨
%     % \[ f'(x)=\lim_{h \to 0} \frac{f(x+h)-f(x)}{h} \]
%     \item 이 정의는 함수의 기울기를 나타내며, 접선의 기울기와 관련됨
%   \end{itemize}

% 제일 쉽고 간편하게 개념을 느끼고 사용하면 됨.
% 크게 세가지로 보여 줄 수 있음. 1. 대수학적으로 보여주기 2.기하학적적으로 보여주기 3.수학적 명제로 증명하기.
% 수학적 명제가 가장 완결성을 가지지만 추상적이기 때문에 '느끼기'힘듦.
% 대수학적으로 보여주고, 기하학적으로 보여주고, 다시 대수학적으로 살펴볼 것.
% 우리의 목적은 수하적으로 완결한 미분을 배우는것이 아니라, 아래 문제를 풀기 위함임.
% 1.효용극대화 2.이윤극대화 3.비용최소화
% 이 문제들은 미분을 통해 풀 수 있음.


\section{도함수와 미분}
\begin{frame}{도함수의 정의로 미분}
  \begin{definition}[도함수]
    실수($\mathbb{R}$)의 어떤 함수 \textcolor{violet}{$f(x)$}가 정의되는 포인트 \textcolor{blue}{\emph{$x_0$}} 에서 \emph{미분가능(differentiable)} 하고, 정의역이 포인트 \textcolor{blue}{\emph{$x_0$}} 를 포함한다면, \textcolor{blue}{\emph{$x_0$}} 에서 \textcolor{red}{미분계수(순간변화율)} \textcolor{red}{\emph{$L$}} 은 \\
    \begin{equation}
      \textcolor{red}{L} = \textcolor{teal}{\lim_{h \to 0} \frac{f(\textcolor{blue}{x_0}+h)-f(\textcolor{blue}{x_0})}{h}}
    \end{equation}
  \end{definition}
  % 
  \begin{align*}
    & f(x) = ax, \quad \text{도함수:} f'(x) = \lim_{h \to 0} \frac{f(x+h)-f(x)}{h} \\
    & \text{도함수에 대입} \rightarrow f'(x) = \lim_{h \to 0} \frac{a(x+h)-ax}{h} \\
    & = \lim_{h \to 0} \frac{ax + ah -ax}{h} = \lim_{h \to 0} \frac{ah}{h} = \lim_{h \to 0} a = a\\
  \end{align*}
  % 
\end{frame}


\begin{frame}{도함수의 정의로 미분}
  \begin{align*}
    & f(x) = ax^2, \quad \text{도함수:} f'(x) = \lim_{h \to 0} \frac{f(x+h)-f(x)}{h} \\
    & \text{도함수에 대입} \rightarrow f'(x) = \lim_{h \to 0} \frac{a(x +h)^2 - ax^2}{h} \\
    & = \lim_{h \to 0} \frac{ax^2 + 2axh + ah^2 - ax^2}{h} \\
    & = \lim_{h \to 0} \frac{2axh + ah^2}{h} = \lim_{h \to 0} 2ax + \lim_{h \to 0} ah = 2ax + a \cdot 0 = 2ax \\
    & \therefore f'(x) = 2ax \\
    & f'(x \mid x=1) = 2a\cdot 1, \quad f'(x \mid x=2) = 4a
  \end{align*}
  % 
\end{frame}

\begin{frame}{도함수의 정의로 미분}
  \begin{align*}
    & f(x) = \frac{1}{x} = x^{-1}, \quad \text{도함수:} f'(x) = \lim_{h \to 0} \frac{f(x+h)-f(x)}{h} \\
    & \text{도함수에 대입} \rightarrow f'(x) = \lim_{h \to 0} \frac{\frac{1}{x+h} - \frac{1}{x}}{h} \\
    & = \lim_{h \to 0} \frac{\frac{x - (x + h)}{x(x+h)}}{h}  = \lim_{h \to 0} \frac{\frac{-h}{x(x+h)}}{h} = \lim_{h \to 0} \frac{-h}{hx(x+h)}   \\
    & = \lim_{h \to 0} \frac{-1}{x(x+h)} = \lim_{h \to 0} \frac{-1}{x^2+hx}   = \frac{-1}{x^2} \\
    & \therefore f'(x) = -\frac{1}{x^2} = {-1}x^{-2} \\
    & f'(x \mid x=1) = -1, \quad f'(x \mid x=2) = -\frac{1}{4}
  \end{align*}
\end{frame}

\begin{frame}{연속성과 미분 가능성}
  \textbf{연속성:}
  \begin{itemize}
    \item 함수 \( f(x) \)가 \( x = a \)에서 연속이려면:
    \[\lim_{x \to a} f(x) = f(a)\]
  \end{itemize}
  
  \bigskip
  \textbf{미분 가능성:}
  \begin{itemize}
    \item 함수 \( f(x) \)가 \( x = a \) 에서 미분 가능하려면:
    \[f'(a) = \lim_{h \to 0} \frac{f(a+h)-f(a)}{h} \quad \text{존재해야 함}\]
    \item 미분 가능성은 \textit{연속성을 포함}합니다.
    \item 연속성이 있다고 해서 \textit{항상 미분 가능하진 않음}.
  \end{itemize}
\end{frame}



\begin{frame}{도함수의 정의로 미분:미분법칙}
  \begin{align*}
    & f(x) = ax^2 \rightarrow f'(x) = 2ax \\
    & f(x) = \frac{1}{x} = x^{-1} \rightarrow  f'(x) = -\frac{1}{x^2} = ({-1})x^{-2}
  \end{align*}
  \vspace{10pt}
  \begin{MyTheorem}{미분법칙}{}
    \phantom{.}\( \frac{d}{dx} x^n = n \cdot x^{n-1} \)\phantom{.}
  \end{MyTheorem}
\end{frame}



\begin{frame}{도함수의 정의로 미분 - 지수함수}
  \begin{align*}
    & f(x) = e^x, \quad \text{도함수:} f'(x) = \lim_{h \to 0} \frac{f(x+h)-f(x)}{h} \\
    & \text{도함수에 대입} \rightarrow f'(x) = \lim_{h \to 0} \frac{e^{x+h} - e^x}{h} = \lim_{h \to 0} \frac{e^x \cdot e^h - e^x}{h} \\
    & = \lim_{h \to 0} \frac{e^x (e^h - 1)}{h} = e^x \cdot \lim_{h \to 0} \frac{e^h - 1}{h} \\
    & e^h \text{의 정의 } e^h = 1 + \frac{h}{1!} + \frac{h^2}{2!} + \cdots \text{ 를 활용하면,}\\
    & = e^x \cdot \lim_{h \to 0} \frac{1 + \frac{h}{1!} + \frac{h^2}{2!} + \cdots - 1}{h}  = e^x \cdot \lim_{h \to 0} (\frac{h}{1!}/h + \frac{h^2}{2!}/h + \cdots) \\
    & = e^x \cdot (1 + 0 + \cdots) = e^x    
  \end{align*}
\end{frame}


\begin{frame}{도함수의 정의로 미분 - 지수함수}
  \scalebox{0.85}{%
  \parbox{\linewidth}{%
  \begin{align*}    
    & f(x) = a^x, \text{도함수에 대입} \rightarrow f'(x) = \lim_{h \to 0} \frac{a^{x+h} - a^x}{h} = \lim_{h \to 0} \frac{a^x \cdot a^h - a^x}{h} \\
    & = \lim_{h \to 0} \frac{a^x (a^h - 1)}{h} = a^x \cdot \lim_{h \to 0} \frac{a^h - 1}{h} \\
    & e^{\ln x} = x \text{의 성질을 이용하여 } a = e^{\ln a} \text{ 양변을 h 거듭제곱한다면, 로그의 성질에 의해 } \\
    & \Rightarrow a^h = e^{h \ln a} \\
    & e \text{의 정의를 활용하기 위해, 양변에 1을빼고 h로 나누면, } \Rightarrow \frac{a^h - 1}{h} = \frac{e^{h \ln a} - 1}{h} \\
    & h\ln a = u \text{ 로 치환하고 정리하면, } h = \frac{u}{\ln a}, \quad \frac{e^{h \ln a} - 1}{h} = \frac{e^u - 1}{\frac{u}{\ln a}} = \frac{e^u - 1}{u} \cdot \ln a \\
    & \text{이때, 극한은 $u$에 대해서로 변함: } \lim_{h \to 0} h\ln a = 0 \text{ 이므로 } u \to 0 \\
    & = a^x \cdot \lim_{u \to 0} \frac{e^u -1}{u} \cdot \ln a = a^x \cdot 1 \cdot \ln a
  \end{align*}
  }%
  }
\end{frame}


\begin{frame}{도함수의 정의로 미분:지수함수 미분법칙}
  \begin{align*}
    & f(x) = e^x \rightarrow f'(x) = e^x \\
    & f(x) = a^x \rightarrow  f'(x) = a^x \cdot \ln a
  \end{align*}
  \vspace{10pt}
  \begin{MyTheorem}{미분법칙}{}
    \phantom{.}\( \frac{d}{dx} e^x = e^x \)\phantom{.} \\
    \phantom{.}\( \frac{d}{dx} a^x = a^x \cdot \ln a \)\phantom{.} \\
  \end{MyTheorem}
\end{frame}




\begin{frame}{도함수의 정의로 미분 - 자연로그}
  \scalebox{0.85}{%
  \parbox{\linewidth}{%
  \begin{align*}
    & f(x) = \ln x, \quad \text{도함수 정의: } f'(x) = \lim_{h \to 0} \frac{\ln(x+h) - \ln x}{h} \\
    & \text{로그의 성질 사용: } \ln(x+h) - \ln x = \ln\left( \frac{x+h}{x} \right) = \ln\left(1 + \frac{h}{x} \right) \\
    & \Rightarrow f'(x) = \lim_{h \to 0} \frac{1}{h} \cdot \ln\left(1 + \frac{h}{x} \right) \\
    & \text{치환: } u = \frac{h}{x} \Rightarrow h = ux, \quad h \to 0 \Leftrightarrow u \to 0 \\
    & \Rightarrow f'(x) = \lim_{u \to 0} \frac{1}{ux} \cdot \ln(1 + u) = \frac{1}{x} \cdot \lim_{u \to 0} \frac{\ln(1+u)}{u} \\
    & \text{Mercator series를 이용해, 전개하면 } \ln(1+u) = u - \frac{u^2}{2} + \frac{u^3}{3} - \frac{u^4}{4} + \cdots \\
    & \lim_{u \to 0} \frac{\ln(1+u)}{u} = \lim_{u \to 0} ( u/u - \frac{u^2}{2} /u + \frac{u^3}{3} /u + \frac{u^4}{4}  /u + \cdots) = 1 - 0 + 0 + \cdots = 1 \text{ 이므로,} \\
    & \therefore f'(x)=\frac{1}{x} \cdot \lim_{u \to 0} \frac{\ln(1+u)}{u} = \frac{1}{x} \cdot 1
  \end{align*}
    }%
    }
\end{frame}


\begin{frame}{도함수의 정의로 미분 - 밑이 \( a \) 인 로그}
  \begin{align*}
    & f(x) = \log_a x = \frac{\ln x}{\ln a}, \quad (\ln a \text{ 는 상수}) \\
    & \text{따라서 } f'(x) = \frac{1}{\ln a} \cdot \frac{d}{dx} \ln x = \frac{1}{\ln a} \cdot \frac{1}{x} \\
    & \therefore f'(x) = \frac{1}{x \ln a}
  \end{align*}
  % \vspace{10pt}
  % \textbf{참고:} \( \log_a x \) 는 자연로그로 표현해 미분할 수 있으며, \(\ln a\) 는 상수입니다.
\end{frame}


\begin{frame}{도함수의 정의로 미분: 로그함수 미분법칙}
  \begin{align*}
    & f(x) = \ln x \rightarrow f'(x) = \frac{1}{x} \\
    & f(x) = \log_a x \rightarrow f'(x) = \frac{1}{x \ln a}
  \end{align*}
  \vspace{10pt}
  \begin{MyTheorem}{미분법칙}{}
    \phantom{.}\( \frac{d}{dx} \ln x = \frac{1}{x} \)\phantom{.} \\[4pt]
    \phantom{.}\( \frac{d}{dx} \log_a x = \frac{1}{x \ln a} \)\phantom{.}
  \end{MyTheorem}
\end{frame}






\begin{frame}{도함수의 정의로 미분 - 곱의 미분법칙}
  \scalebox{0.85}{%
  \parbox{\linewidth}{%
  \begin{align*}
    & \text{함수 } \textcolor{blue}{f(x)}, \textcolor{red}{g(x)} \text{ 에 대해 } h(x) = \textcolor{blue}{f(x)}\textcolor{red}{g(x)} \text{ 라 하자} \\
    & h'(x) = \lim_{h \to 0} \frac{\textcolor{blue}{f(x+h)}\textcolor{red}{g(x+h)} - \textcolor{blue}{f(x)}\textcolor{red}{g(x)}}{h} \\
    \\
    & \textcolor{gray}{\text{두 항의 차를 직접 다루기 어렵기 때문에, 더하고 빼기}} \\
    & = \lim_{h \to 0} \frac{\textcolor{blue}{f(x+h)}\textcolor{red}{g(x+h)} - \textcolor{blue}{f(x)}\textcolor{red}{g(x+h)} + \textcolor{blue}{f(x)}\textcolor{red}{g(x+h)} - \textcolor{blue}{f(x)}\textcolor{red}{g(x)}}{h} \\
    & = \lim_{h \to 0} \left[ \frac{\textcolor{blue}{f(x+h)} - \textcolor{blue}{f(x)}}{h} \cdot \textcolor{red}{g(x+h)} + \textcolor{blue}{f(x)} \cdot \frac{\textcolor{red}{g(x+h)} - \textcolor{red}{g(x)}}{h} \right] \\
    \\
    & \text{극한은 각각 따로 가능하므로:} \\
    & = \left( \lim_{h \to 0} \frac{\textcolor{blue}{f(x+h)} - \textcolor{blue}{f(x)}}{h} \right) \cdot \left( \lim_{h \to 0} \textcolor{red}{g(x+h)} \right) + \textcolor{blue}{f(x)} \cdot \left( \lim_{h \to 0} \frac{\textcolor{red}{g(x+h)} - \textcolor{red}{g(x)}}{h} \right) \\
    & = \textcolor{blue}{f'(x)} \cdot \textcolor{red}{g(x)} + \textcolor{blue}{f(x)} \cdot \textcolor{red}{g'(x)} \\
    \\
    & \therefore \boxed{(\textcolor{blue}{f} \cdot \textcolor{red}{g})'(x) = \textcolor{blue}{f'(x)}\textcolor{red}{g(x)} + \textcolor{blue}{f(x)}\textcolor{red}{g'(x)}}
  \end{align*}
  }%
  }
\end{frame}

\begin{frame}{예제}
  콥더글라스
\end{frame}



\begin{frame}{도함수의 정의로 미분 - 합성함수의 미분법칙 (Chain Rule)}
  \scalebox{0.85}{%
  \parbox{\linewidth}{%
  \begin{align*}
    & \text{합성함수 } h(x) = \textcolor{blue}{f(}\textcolor{red}{g(x)}\textcolor{blue}{)} \text{ 에 대해 도함수를 정의로 구함} \\
    & h'(x) = \lim_{h \to 0} \frac{\textcolor{blue}{f(}\textcolor{red}{g(x+h)}\textcolor{blue}{)} - \textcolor{blue}{f(}\textcolor{red}{g(x)}\textcolor{blue}{)}}{h} \\
    % \\
    % & \textcolor{gray}{\text{(중간 단계: 분모에 } \textcolor{red}{g(x+h) - g(x)} \textcolor{gray}{ 를 곱하고 나누기)}} \\
    & = \lim_{h \to 0} \left[ \frac{\textcolor{blue}{f(}\textcolor{red}{g(x+h)}\textcolor{blue}{)} - \textcolor{blue}{f(}\textcolor{red}{g(x)}\textcolor{blue}{)}}{\textcolor{red}{g(x+h) - g(x)}} \cdot \frac{\textcolor{red}{g(x+h) - g(x)}}{h} \right] \\
    & \text{앞부분은 마치 } \frac{\Delta f(g(x))}{\Delta g(x)} \text{ 인데 $\Delta \to 0 $ 이면 미분의 정의} \\
    & \text{극한을 각각 분리하면:} \\
    & = \left( \lim_{\textcolor{red}{u} \to \textcolor{red}{g(x)}} \frac{\textcolor{blue}{f(}\textcolor{red}{u}\textcolor{blue}{)} - \textcolor{blue}{f(}\textcolor{red}{g(x)}\textcolor{blue}{)}}{\textcolor{red}{u - g(x)}} \right) \cdot \left( \lim_{h \to 0} \frac{\textcolor{red}{g(x+h)} - \textcolor{red}{g(x)}}{h} \right) \\
    & = \textcolor{blue}{f'}(\textcolor{red}{g(x)}) \cdot \textcolor{red}{g'(x)} \\
    % \\
    & \therefore \boxed{( \textcolor{blue}{f} \circ \textcolor{red}{g} )'(x) = \textcolor{blue}{f'}(\textcolor{red}{g(x)}) \cdot \textcolor{red}{g'(x)}}
  \end{align*}
  }%
  }
\end{frame}


\begin{frame}{비교정태분석 (Comparative Statics)}
  \textbf{비교정태분석이란?}
  \begin{itemize}
    \item 매개변수 변화에 따른 두 균형 상태를 비교하는 분석.
    \item 외생변수 또는 매개변수가 변할 때 내생변수가 어떻게 조정되는지를 연구.
  \end{itemize}
  
  \bigskip
  \textbf{비교정태분석의 종류}
  \begin{itemize}
    \item \textbf{질적 분석}: 변화의 방향만 분석 (증가 또는 감소).
    \item \textbf{양적 분석}: 변화의 방향과 크기를 함께 분석.
  \end{itemize}
  \end{frame}
  
  
\begin{frame}{미분의 경제학적 해석}
  \textbf{한계 개념들:}
  
  미분은 한계 경제 지표를 정의하는 데 핵심적인 도구입니다.
  \bigskip
  \begin{itemize}
    \item \textbf{한계비용 (MC)}: 
    \[MC = \frac{dC(Q)}{dQ}\]
    \item \textbf{한계수입 (MR)}:
    \[MR = \frac{dR(Q)}{dQ}\]
    \item \textbf{한계소비성향 (MPC)}:
    \[MPC = \frac{dC(Y)}{dY}\]
  \end{itemize}
  MRS, MRTS 
\end{frame}



\section{생산함수: 콥더글라스}

\section{효용함수: 분수함수수}

\section{완전대체제: 선형함수}

\section{완전보완제: 레온티예프}




\begin{frame}{예제 - 쉬프트된 함수들}
  \begin{align*}
    f(x) = (x+1)^2
  \end{align*}
\end{frame}




\begin{frame}
  
\end{frame}



\end{document}

