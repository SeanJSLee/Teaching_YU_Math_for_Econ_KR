\documentclass[aspectratio=169]{beamer}
\usepackage{lectureppt} % Load your custom style



\title{수리경제학}
\subtitle{중간고사 풀이}
% \subtitle{직관적 이해와 활용 테스트트}
\author{이재석}
\date{2025-04-29}

\begin{document}

\begin{frame}
  \titlepage
\end{frame}



\begin{frame}
  1. 계수행렬 \( A \)가 다음과 같이 주어지고, 행렬식 \( |A| = -6 \)이다. (20점)
  \[
    A = \begin{bmatrix}
    8 & 1 & 3 \\
    4 & 0 & 1 \\
    6 & 0 & 3
    \end{bmatrix}
  \]
  (1) 1행과 2행을 서로 교환하면 행렬식은 얼마인가? (5점) \\
  \hspace{30pt} \textbf{\emph{6}} \\
  (열이나 행을 홀수번 교환시 행렬식 부호 바뀜) \\
\end{frame}


\begin{frame}
  (2) 
  \[
    A = \begin{bmatrix}
    8 & 2 & 3 \\
    4 & 0 & 1 \\
    6 & 0 & 3
    \end{bmatrix}
  \]
  의 행렬식은 얼마인가? (5점) \\
  \hspace{30pt} \textbf{\emph{-12 }} \\
  $|A'| = a_{11} |C_{11}| + 2 * a_{12} |C_{12}| + a_{13} |C_{13}| = |A| + a_{12} |C_{12}| = -6 + (1)(-1)^{1+2} (4\cdot 3 - 1 \cdot 6) = -6 -6$ 

\end{frame}



\begin{frame}
  A를 확장한 다음 행렬
  \[
    A = \begin{bmatrix}
    8 & 1 & 3 & 0 \\
    4 & 0 & 1 & 0 \\
    6 & 0 & 3 & 0 \\
    0 & 0 & 0 & 2
    \end{bmatrix}
  \]
  의 행렬식은 얼마인가? (10점) \\
  \hspace{30pt} \textbf{\emph{ $(-1) 2 (-1)^{1+4} (-6) = -12$  }} \\

  1행과 4행을 스왑한다면, $|(C_{14})'| = (-1)^{1+4} \begin{vmatrix}
    4 & 0 & 1 \\
    6 & 0 & 3 \\
    8 & 2 & 3 
    \end{vmatrix}$
    이때 다시, 1,3 스왑 그리고 2,3 을 스왑한다면 A와 같아짐. 그리고 두번 스왑했으므로 행렬식도 -6 으로 동일. 따라서 $M_{14} = -6$ \\
    A의 행렬식은 스왑한 행렬식의 부호가 바뀐 것. \\
    $|A| = (-1) |A'| = (-1) \cdot 2 * |(C_{14})'| = (-1) 2 (-1)^{1+4} (-6) = -12$


\end{frame}


\begin{frame}
  2. 수많은 방정식으로 구성된 국민소득 모형을 생각해보자. 이를 행렬과 벡터로 표시했더니 \( Ax = d \)로 단순화되었다.
이 선형방정식 체계의 해 \( x^* \)를 아래 기호를 이용해 구성하라. (10점)

(A, A', $|A|$, C, M, adj A, d)

  $ x^* = A^{(-1)}d = \frac{1}{|A|} adj(A) d $

\end{frame}


\begin{frame}
  3. 다음 행렬 \( A \)에 대해:
    \[
    A = \begin{bmatrix}
        6 &  3  &  1 &  0    \\
        1 &  4  & -2 &  0   \\
        4 & -1  &  5 &  0    \\
        0 &  0  &  0 &  1
    \end{bmatrix}
    \]
    의 $|A|$ 를 계산하고 특이성(singularity)을 검정하라. (20점) \\

    1,4를 스왑한 $A'$ 라면, \\
    $|A| = (-1) |A'| = (-1) \cdot 1 \cdot (-1)^{1+4} |M'_{14}| = 
    \begin{vmatrix}
      1 &  4  & -2    \\
      4 & -1  &  5     \\
      6 &  3  &  1    
  \end{vmatrix} 
  = 1\cdot (-1-15) - 4 \cdot (4-30) + (-2) \cdot (12+6) = 52
    $ \\
    행렬식이 0이 아니므로, 비특이행렬.

    

\end{frame}


\begin{frame}
  4. 
  \[
  A = \begin{bmatrix}
  0 & 1 & 0 \\
  1 & 0 & 0 \\
  0 & 0 & 1
  \end{bmatrix}
  \]
  의 역행렬을 구하라. (10점) \\
  
  $|A| = -1$


  \[
  C = \begin{bmatrix}
    (-1)^{1+1}\cdot 0 & (-1)^{1+2}\cdot 1 & (-1)^{1+3}\cdot 0 \\
    (-1)^{2+1}\cdot 1 & (-1)^{2+2}\cdot 0 & (-1)^{2+3}\cdot 0 \\
    (-1)^{3+1}\cdot 0 & (-1)^{3+2}\cdot 0 & (-1)^{3+3}\cdot (-1) 
  \end{bmatrix}
  = \begin{bmatrix}
    0  & -1 &  0 \\
    -1 &  0 &  0 \\
    0  &  0 & -1 
  \end{bmatrix}
  \]

  $
  A^{-1}= \begin{bmatrix}
    0  & -1 &  0 \\
    -1 &  0 &  0 \\
    0  &  0 & -1 
  \end{bmatrix}
  $


\end{frame}



\begin{frame}
  5. 다음의 선형방정식 체계가 주어졌다. 주어진 문제에 답하라. (30점)

\[
\begin{aligned}
6x + 3y + z &= 22 \\
x + 4y - 2z &= 12 \\
4x - y + 5z &= 10
\end{aligned}
\]


(1) 주어진 연립방정식을 행렬과 벡터로 나타내어라. (5점)

$
\begin{bmatrix}
  6  &  3  &  1 \\
  1  &  4  & -2 \\
  4  &  -1 &  5 
\end{bmatrix}
\begin{bmatrix}
  x \\
  y \\
  z 
\end{bmatrix}
=
\begin{bmatrix}
  22 \\
  12 \\
  10 
\end{bmatrix}
$




\end{frame}




\begin{frame}
  (2) 계수행렬의 행렬식 \( |A| \)을 라플라스 전개를 이용해 구하라. (5점)

  \[
  |A| = 6 \cdot (20-2) - 3 \cdot (5+8) +1 \cdot (-1-16) = 52
  \]

  (3) \( \text{adj}(A) \)를 구하라. (10점)
  \[
    \text{adj}(A) = 
    \begin{bmatrix}
      (20-2) & -(5+8) & (-1-16) \\
      -(15+1) & (30-4) & -(-6-12) \\
      (-6-4) & -(-12-1) & (24 - 3)
    \end{bmatrix}'
    =\begin{bmatrix}
      18 & -16 & -10 \\
      -13 & 26 & 13 \\
      -17 & 18 & 21
    \end{bmatrix}
  \]


\end{frame}



\begin{frame}
  (4) 행렬 \( A \)의 역행렬을 구하라. (5점)

  \[
    A^{-1} = 
    \frac{1}{52}
    \begin{bmatrix}
      18 & -16 & -10 \\
      -13 & 26 & 13 \\
      -17 & 18 & 21
    \end{bmatrix}
  \]
  
  (5) 위의 결과를 이용해 \( x, y, z \)의 값을 구하라. (5점)  
  \[
    \begin{bmatrix}
      x \\
      y \\
      z 
    \end{bmatrix}
     = 
    \frac{1}{52}
    \begin{bmatrix}
      18 & -16 & -10 \\
      -13 & 26 & 13 \\
      -17 & 18 & 21
    \end{bmatrix}
    \begin{bmatrix}
      22 \\
      12 \\
      10 
    \end{bmatrix}
  \]
  $x = \frac{1}{52}(18*22 - 16*12 - 10*10) = \frac{104}{52} = 2$
  $y = \frac{1}{52}(-13*22 + 26*12 + 13*10) = \frac{156}{52} = 3$
  $z = \frac{1}{52}(-17*22 + 18*12 + 21*10) = \frac{52}{52} = 1$
\end{frame}




\begin{frame}
  6. '크래머의 법칙(Cramer's Rule)'을 이용하여 \( x, y, z \)의 해를 구하라. 풀이 과정을 요약해 적을 것. (20점)

  \[
  \begin{aligned}
  6x + 3y - 2z &= 7 \\
  7x - y - z &= 0 \\
  10x - 2y + z &= 8
  \end{aligned}
  \]

  $
  \begin{bmatrix}
    6  &  3  & -2 \\
    7  & -1  & -1 \\
   10  & -2 &  1 
  \end{bmatrix}
  \begin{bmatrix}
    x \\
    y \\
    z 
  \end{bmatrix}
  =
  \begin{bmatrix}
    7 \\
    0 \\
    8 
  \end{bmatrix}
  , \quad 
  |A| = -61
  , \quad 
  x = \frac{1}{|A|} \begin{vmatrix}
    7  &  3  & -2 \\
    0  & -1  & -1 \\
    8  & -2 &  1 
  \end{vmatrix} = \frac{-61}{-61} = 1 
  $

  \[ y = \frac{1}{|A|} \begin{vmatrix}
     6 & 7  & -2 \\
     7 & 0  & -1 \\
    10 & 8  &  1 
  \end{vmatrix} = \frac{-183}{-61} = 3 
  , \quad 
  z = \frac{1}{|A|} \begin{vmatrix}
     6 &  3  & 7  \\
     7 & -1  & 0  \\
    10 & -2  &  8 
  \end{vmatrix} = \frac{-244}{-61} = 4 
  \]


\end{frame}



\begin{frame}
  7. \( y = (3x^3 + 4x^2 + 3x + 4)^{1000} \) 에 대해 다음을 구하시오. (20점)


    (1)
    \[
    \frac{dy}{d(3x^3 + 4x^2 + 3x + 4)} = 1000(3x^3 + 4x^2 + 3x + 4)^{999}
    \]

    (2)
    \[
    \frac{dy}{dx} = 1000(3x^3 + 4x^2 + 3x + 4)^{999} (9x^2 + 8x + 3)
    \]
\end{frame}




\end{document}