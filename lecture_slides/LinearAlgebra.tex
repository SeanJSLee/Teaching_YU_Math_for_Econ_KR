\documentclass[aspectratio=169]{beamer}
\usepackage{lectureppt} % Load your custom style
% \usepackage[utf8]{inputenc}



\title{수리경제학}
\subtitle{리뷰: 행렬}
\author{이재석}
\date{2025-04-15 \\ (updated:\today)}


\begin{document}

\begin{frame}
  \titlepage
\end{frame}

\begin{frame}{목차}
  \tableofcontents
\end{frame}


\section{행렬의 존재 이유}
\begin{frame}{'인공'적인 행렬}
  \begin{itemize}
    \item 라이프니츠 이전, 여러가지 '행렬'같은 기법들이 있었으나..
    \item 1700-1710년 라이프니츠가 선형 연립방정식의 '계수'들을 행과 열형태로 표현하고 연립방정식의 해가 존재하는 조건을 보여줌. \\ %계수들의 행과 열을 'Matrix'라고 이름 붙임. \\
    * 역행렬을 구하기 위한 조건. 예를 들어 $ \left| A \right| \neq 0$
    \item 1750년 크레이머가 Cramer's Rule 을 발표.
    \item 1850년 실베스터는 행렬의 성질을 'minors'라는 특별한 행렬식으로 정리함. \\
      '마치 마법처럼' 연립방정식의 해를 구하는 방법이었기에, 실베스터는 라틴어로 '근원'을 의미하는 'Matrix'라는 이름을 붙임.
  \end{itemize}
  \vspace{10pt}
  이처럼, 매트릭스는 연립방정식의 해를 찾기 위해 '만들어진' 것이기에, 직관적이지 않아서 익숙해지기가 너무 힘들었다. \\
  하지만, 어쩌면 '발견'한 것 처럼, 마법처럼 효율적으로 해를 찾을 수 있다. 컴퓨터도 행렬로 해를 찾음.
\end{frame}


\begin{frame}{경제학에서 행렬을 배우는 이유: $\beta=(X'X)^{-1}X'Y$}
  \begin{itemize}
    \item 경제학에서 행렬을 공부하는 이유는 회귀분석을 사용하기 위함.
    \item 회귀분석의 시작은 OLS라 불리는 선형회귀모형.
    \item OLS는 연립방정식의 해를 구하는 것과 같음.
    \begin{itemize}
      \item 경제모델: $Y = \beta_{K}\cdot X_K$ + $\beta_{L}\cdot X_L + \varepsilon(\text{오차항})$
      \item $\beta=(X'X)^{-1}X'Y$
    \end{itemize}
    \item $\beta$를 구하기 위해서:
    \begin{itemize}
      \item 행렬곱, 역행렬( 역행렬 찾기, 비 특이행렬 확인(행렬식$\neq 0$))
      \item 라플라스 전개, minor, cofactor
      \item 크레이머 룰.
    \end{itemize}
  \end{itemize}
\end{frame}



\begin{frame}{행렬의 연산 }
  행렬에는 '자연스러운' 사칙연산을 포함하여, '특별한 규칙을 갖는' 다양한 연산이 존재함. \\
  \begin{itemize}
    \item 행렬의 덧셈, 뺄셈, 스칼라 곱 
    \item 행렬의 곱셈 
    \item 행렬의 전치
    \item 행렬식
    % \item inner product, outer product
    \item minors, cofactors
    \item $\cdots$
  \end{itemize}
  그저, 필요한 규칙들을 쓰임에 맞게 조합해서 사용한다고 생각하면 됨.
\end{frame}






\begin{frame}{행렬과 연립방정식}
  \begin{columns}
    \begin{column}{0.5\textwidth}
      m by n 행렬 $A$ (m행 n열): \\
      \[
        A_{\textcolor{blue}{m}\times \textcolor{red}{n}} =
        \left[ {\begin{array}{cccc}
          a_{\textcolor{blue}{1}\textcolor{red}{1}} & a_{\textcolor{blue}{1}\textcolor{red}{2}} & \cdots & a_{\textcolor{blue}{1}\textcolor{red}{n}}\\
          a_{\textcolor{blue}{2}\textcolor{red}{1}} & a_{\textcolor{blue}{2}\textcolor{red}{2}} & \cdots & a_{\textcolor{blue}{2}\textcolor{red}{n}}\\
          \vdots & \vdots & \ddots & \vdots\\
          a_{\textcolor{blue}{m}\textcolor{red}{1}} & a_{\textcolor{blue}{m}\textcolor{red}{2}} & \cdots & a_{\textcolor{blue}{m}\textcolor{red}{n}}\\
        \end{array} } \right]
      \]    
    \end{column}
    % 
    \begin{column}{0.5\textwidth}
      라이프니츠 처럼 연립방정식을 행렬로 표현하면,\\
      \begin{alignat*}{4}
        2x & {}+{} &  y & {}+{} & 3z & {}={} & 10 \\
         x & {}+{} &  y & {}+{} &  z & {}={} &  6 \\
         x & {}+{} & 3y & {}+{} & 2z & {}={} & 13
      \end{alignat*}
      \centering{$\Downarrow$}\\
      $Ax = d$
      \[      
      \begin{bmatrix}
        2 & 1 & 3 \\
        1 & 1 & 1 \\
        1 & 3 & 2 
      \end{bmatrix}
      \begin{bmatrix}
        x \\ y \\ z 
      \end{bmatrix}
      =
      \begin{bmatrix}
        10 \\ 6 \\ 13 
      \end{bmatrix}
      \]
    \end{column}
  \end{columns}
\end{frame}



% 
\section{행렬 $^\text{(자연스러운)}$ 연산 }

\begin{frame}{행렬의 연산 - 덧셈, 뺄셈, 스칼라 곱}
  \begin{columns}
    \begin{column}{0.5\textwidth}
      \begin{align*}
        A &= \begin{bmatrix} 2 & 3 \\ 1 & 4 \end{bmatrix}, \\
        B &= \begin{bmatrix} 5 & 1 \\ 0 & -2 \end{bmatrix} \\
        \vspace{10pt}
        \phantom{blank} \\
        A + B &= \begin{bmatrix} 7 & 4 \\ 1 & 2 \end{bmatrix} \\
        2A &= \begin{bmatrix} 4 & 6 \\ 2 & 8 \end{bmatrix}
      \end{align*}
    \end{column}
    % 
    \begin{column}{0.5\textwidth}
      \begin{itemize}
        \item 두 행렬 \( A \)와 \( B \)가 같은 차원일 때만 덧셈과 뺄셈 가능
        \item 스칼라 곱: 행렬의 각 원소에 같은 수를 곱함
      \end{itemize}
    \end{column}
  \end{columns}
\end{frame}





\section{행렬곱}
\begin{frame}{행렬곱의 조건 및 연산 규칙}
\begin{itemize}
    \item 행렬 A (m×\textcolor{teal}{p})과 B (\textcolor{teal}{p}×n) 조건: 앞 행렬의 \textcolor{teal}{열 수} = 뒤 행렬의 \textcolor{teal}{행 수} \\
       (3x\textcolor{teal}{1})(\textcolor{teal}{1}x3): 가능! $\Rightarrow$ 곱셈 결과 m×n 행렬 \\
       (3x\textcolor{teal}{1})(\textcolor{teal}{3}x1): 불가능! $\Rightarrow$ 행렬곱에서 교환법칙이 (항상)성립되지 않는 이유! ($AB \neq BA$)\\
    \item 아래 행렬곱 결과의 차원(mxn)은?
       \\ (3x3)(3x2) =
       \\ (5x5)(5x5) =
       \\ (3x5)(5x2) =
       \\ (4x2)(3x2) =
       \\ (2x100)(100x2) =
\end{itemize}
\end{frame}


\begin{frame}{행렬곱의 계산}
\[
\begin{bNiceMatrix}
  \Block[draw,fill=red!5,rounded-corners]{1-4}{}
  a_{\textcolor{red}{1}1} & a_{\textcolor{red}{1}2} & \cdots & a_{\textcolor{red}{1}p} \\
  a_{\textcolor{red}{2}1} & a_{\textcolor{red}{2}2} & \cdots & a_{\textcolor{red}{2}p} \\
  \vdots & \vdots & \ddots & \vdots \\
  a_{\textcolor{red}{m}1} & a_{\textcolor{red}{m}2} & \cdots & a_{\textcolor{red}{m}p}
\end{bNiceMatrix}
\cdot
\begin{bNiceMatrix}
  \Block[draw,fill=red!15,rounded-corners]{4-1}{}
  b_{1\textcolor{blue}{1}} & 
  \Block[draw,fill=blue!15,rounded-corners]{4-1}{}
  b_{1\textcolor{blue}{2}} & \cdots & 
  \Block[draw,fill=green!15,rounded-corners]{4-1}{}
  b_{1\textcolor{blue}{n}} \\
  b_{2\textcolor{blue}{1}} & b_{2\textcolor{blue}{2}} & \cdots & b_{2\textcolor{blue}{n}} \\
  \vdots & \vdots & \ddots & \vdots \\
  b_{p\textcolor{blue}{1}} & b_{p\textcolor{blue}{2}} & \cdots & b_{p\textcolor{blue}{n}}
\end{bNiceMatrix}
=
\begin{bNiceMatrix}
  \Block[draw,fill=red!15,rounded-corners]{1-1}{}
  c_{\textcolor{red}{1}\textcolor{blue}{1}} & 
  \Block[draw,fill=blue!15,rounded-corners]{1-1}{}
  c_{\textcolor{red}{1}\textcolor{blue}{2}} & \cdots & 
  \Block[draw,fill=green!15,rounded-corners]{1-1}{}
  c_{\textcolor{red}{1}\textcolor{blue}{n}} \\
  c_{\textcolor{red}{2}\textcolor{blue}{1}} & c_{\textcolor{red}{2}\textcolor{blue}{2}} & \cdots & c_{\textcolor{red}{2}\textcolor{blue}{n}} \\
  \vdots & \vdots & \ddots & \vdots \\
  c_{\textcolor{red}{m}\textcolor{blue}{1}} & c_{\textcolor{red}{m}\textcolor{blue}{2}} & \cdots & c_{\textcolor{red}{m}\textcolor{blue}{n}}
\end{bNiceMatrix}
\]
\vspace{1em}
\begin{itemize}
  \item \( c_{\textcolor{red}{m}\textcolor{blue}{n}} = \sum_{k=1}^{p} a_{\textcolor{red}{m}k} b_{k\textcolor{blue}{n}} \)
  \begin{itemize}
    \item $ c_{\textcolor{red}{1}\textcolor{blue}{1}} = a_{\textcolor{red}{1}{1}}b_{{1}\textcolor{blue}{1}} + a_{\textcolor{red}{1}{2}}b_{{2}\textcolor{blue}{1}} + \cdots + a_{\textcolor{red}{1}{p}}b_{{p}\textcolor{blue}{1}} $
    \item $ c_{\textcolor{red}{m}\textcolor{blue}{n}} = a_{\textcolor{red}{m}{1}}b_{{1}\textcolor{blue}{n}} + a_{\textcolor{red}{m}{2}}b_{{2}\textcolor{blue}{n}} + \cdots + a_{\textcolor{red}{m}{p}}b_{{p}\textcolor{blue}{n}} $
  \end{itemize}
  % \item 행렬곱이 정의되려면 첫 번째 행렬의 열 수 = 두 번째 행렬의 행 수 (\( p \))
\end{itemize}
\end{frame}



\begin{frame}{행렬곱의 계산: 연습}
\begin{itemize}
    \item \(  \begin{bmatrix} a & b \\ c & d \end{bmatrix}
              \begin{bmatrix} e & f \\ g & h \end{bmatrix} \)
    \vspace{10pt}
    \item \(  \begin{bmatrix} a & b \end{bmatrix}
              \begin{bmatrix} e \\ g \end{bmatrix} \)
    \vspace{10pt}
    \item \(  \begin{bmatrix} a  \\ c  \end{bmatrix}
              \begin{bmatrix} e & f \end{bmatrix} \)
    \vspace{10pt}
    \item \(  \begin{bmatrix} 1 & 0 & 0 \\ 0 & 1 & 0 \\ 0 & 0 & 1 \end{bmatrix}
              \begin{bmatrix} a & b & c \\ d & e & f \\ g & h & i \end{bmatrix} \)
    \vspace{10pt}
\end{itemize}
\end{frame}


\section{행렬식}
\begin{frame}{행렬식}
  \begin{block}{행렬식 (Determinant)}
    정사각행렬 \( A \)의 행렬식 \( \det(A) \) 또는 \( |A| \)은 $A$ 행렬의 특성을 나타내는 값. 각각의 변수들의 관계를 하나의 숫자로 나타낸 것. \\
    (마치 2차방정식 판별식 처럼, 역행렬이 존재하기 위해서는 0이 아닌 값이 필요)
  \end{block}
  \vspace{10pt}
  행렬의 차원에 따라, 행렬식을 계산 하는 법이 다름.
  \begin{itemize}
    \item 1x1 행렬: \( |a_{11}| = a \)
    \item 2x2 행렬: \( \begin{vmatrix} a & b \\ c & d \end{vmatrix} = ad-bc \)
  \end{itemize}
\end{frame}


\begin{frame}{행렬식: 3x3 행렬}
  \begin{align*}
    & \begin{vNiceMatrix} 
        a_{11} & a_{12} & a_{13} \\ 
        a_{21} & a_{22} & a_{23} \\ 
        a_{31} & a_{32} & a_{33} 
      \end{vNiceMatrix}  \\
    \Rightarrow & \begin{vNiceMatrix} 
        \Block[draw,fill=red!15,rounded-corners]{1-1}{}
        a_{11} & a_{12} & a_{13} \\ 
        a_{21} & 
        \Block[draw,fill=red!15,rounded-corners]{2-2}{}
        a_{22} & a_{23} \\ 
        a_{31} & a_{32} & a_{33} 
        \CodeAfter
          \tikz \draw [red] (1.5 -|1.2) -- (1.5 -|3.8) ; % horizontal (x,y)
          \tikz \draw [red] (1.2 -|1.5) -- (3.8 -|1.5) ; % vertical
      \end{vNiceMatrix}
      -
      \begin{vNiceMatrix} 
        a_{11} & 
        \Block[draw,fill=blue!15,rounded-corners]{1-1}{}
        a_{12} & a_{13} \\ 
        \Block[draw,fill=blue!15,rounded-corners]{2-1}{}
        a_{21} & 
        a_{22} & 
        \Block[draw,fill=blue!15,rounded-corners]{2-1}{}
        a_{23} \\ 
        a_{31} & a_{32} & a_{33} 
        \CodeAfter
          \tikz \draw [red] (1.5 -|1.2) -- (1.5 -|3.8) ; % horizontal (x,y)
          \tikz \draw [red] (1.2 -|2.5) -- (3.8 -|2.5) ; % vertical
      \end{vNiceMatrix}
      +
      \begin{vNiceMatrix} 
        a_{11} & a_{12} & 
        \Block[draw,fill=green!15,rounded-corners]{1-1}{}
        a_{13} \\
        \Block[draw,fill=green!15,rounded-corners]{2-2}{} 
        a_{21} & a_{22} & a_{23} \\ 
        a_{31} & a_{32} & a_{33} 
        \CodeAfter
          \tikz \draw [red] (1.5 -|1.2) -- (1.5 -|3.8) ; % horizontal (x,y)
          \tikz \draw [red] (1.2 -|3.5) -- (3.8 -|3.5) ; % vertical
      \end{vNiceMatrix} \\
      =\, & \tcboxmath[colback=red!15, colframe=red!50!black, arc=1pt, boxrule=0.2pt]
      {a_{11} \cdot (a_{22}a_{33} - a_{23}a_{32})} 
        - \tcboxmath[colback=blue!15, colframe=blue!50!black, arc=1pt, boxrule=0.2pt]{a_{12} \cdot (a_{21}a_{33} - a_{23}a_{31})} \\
        & + \tcboxmath[colback=green!15, colframe=green!50!black, arc=1pt, boxrule=0.2pt]{a_{13} \cdot (a_{21}a_{32} - a_{22}a_{31})}
  \end{align*}
\end{frame}


\begin{frame}{행렬식: 3x3 행렬 연습}
  % \begin{align*}
  %   & \begin{vNiceMatrix} 
  %       a_{11} & a_{12} & a_{13} \\ 
  %       a_{21} & a_{22} & a_{23} \\ 
  %       a_{31} & a_{32} & a_{33} 
  %     \end{vNiceMatrix}  \\
  % \end{align*}
\end{frame}


\begin{frame}{행렬식: 3x3 행렬 연습}
  \begin{columns}
    \begin{column}{0.5\textwidth}
      \begin{align*}
        % & 
        \begin{bNiceMatrix}
          2 & 1 & 3 \\
          4 & 5 & 6 \\
          7 & 8 & 9 
        \end{bNiceMatrix} 
        = -9 
        \\[1em]
        % 
        \begin{bmatrix}
          -7 & 0 & 3 \\
           9 & 1 & 4 \\
           0 & 6 & 5 
        \end{bmatrix}
        = 295
      \end{align*}
    \end{column}
    \begin{column}{0.5\textwidth}
      \begin{align*}
        % & 
        \begin{bNiceMatrix}
          8 & 1 & 3 \\
          4 & 0 & 1 \\
          6 & 0 & 3  
        \end{bNiceMatrix} 
        \\[1em]
        % 
        \begin{bmatrix}
          1 & 2 & 3 \\
          4 & 7 & 5 \\
          3 & 6 & 9 
        \end{bmatrix}
      \end{align*}
    \end{column}
  \end{columns}
\end{frame}


\begin{frame}{행렬식: 3x3 행렬 연습 (스왑)}
  \begin{columns}
    \begin{column}{0.5\textwidth}
      \begin{block}{행이나 열을 교환했을때, 행렬식의 부호가 바뀜}
        $|A| = k$ 일때, A의 행, 혹은 열을 홀수번 교환한다면, $-k$가 됨.
      \end{block}
    \end{column}
    \begin{column}{0.5\textwidth}
      \begin{align*}
        % & 
        \begin{bNiceMatrix}
          % \CodeBefore
          % \tikz \draw [fill=blue!15] 
          %     (1 -|1) |- 
          %     (4 -|2) |- 
          %     cycle ;
          % \tikz \draw [fill=green!15] 
          %     (1 -|3) |- 
          %     (4 -|4) |- 
          %     cycle ;
          % \Body
          0 & 1 & 3 \\
          2 & 5 & 7 \\
          3 & 0 & 1
        \end{bNiceMatrix} 
        \\[1em]
        % 
        \begin{bNiceMatrix}
          \CodeBefore
          \tikz \draw [fill=green!15] 
              (1 -|1) |- 
              (4 -|2) |- 
              cycle ;
          \tikz \draw [fill=blue!15] 
              (1 -|3) |- 
              (4 -|4) |- 
              cycle ;
          \Body
          3 & 1 & 0 \\
          7 & 5 & 2 \\
          1 & 0 & 3
        \end{bNiceMatrix}
        \\[1em]
        % 
        \begin{bNiceMatrix}
          \CodeBefore
          \tikz \draw [fill=green!15] 
              (1 -|1) |- 
              (4 -|2) |- 
              cycle ;
          \tikz \draw [fill=blue!15] 
              (1 -|2) |- 
              (4 -|3) |- 
              cycle ;
          \tikz \draw [fill=red!15] 
              (1 -|3) |- 
              (4 -|4) |- 
              cycle ;
          \Body
          3 & 0 & 1  \\
          7 & 2 & 5  \\
          1 & 3 & 0 
        \end{bNiceMatrix}
      \end{align*}
    \end{column}
  \end{columns}
\end{frame}



\begin{frame}{행렬식: nxn 행렬 $\Rightarrow$ 라플라스 전개}
  \begin{itemize}
    \begin{block}{라플라스 전개}
      행렬식을 '부분 행렬식(cofactor)'으로 나누어 계산
      \begin{align*}
        |A| & = a_{11}\cdot |C_{11}| + a_{12}\cdot |C_{12}| + \cdots +a_{1n}\cdot |C_{1n}| \\
            & = a_{11}\cdot |C_{11}| + a_{21}\cdot |C_{21}| + \cdots +a_{m1}\cdot |C_{m1}|, \\
        \text{cofactor}: & |C_{ij}| = (-1)^{i+j} |M_{ij}| ,\quad i:\text{행, } j:\text{열}
      \end{align*}
      % cofactor: $ |C_{ij}| = (-1)^{i+j} |M_{ij}| ,\quad i:\text{행, } j:\text{열} $
    \end{block}
    % \item 라플라스 전개는 
    \item cofactor: 'minor'에 행과 열의 위치에 따라 부호를 붙인 것. *cofactor는 행렬식이므로 '값'.
      % $cofactor: |C_{ij}| = (-1)^{i+j} |M_{ij}| ,\quad i:\text{행, } j:\text{열}$
    \item 'minor'는 행렬에서 특정 행과 열을 제거한 나머지의 행렬식.
    %    이전 3x3 행렬식 계산에서, minor $M_{11}$ 는 아래.
    % \begin{align*}
    %   \begin{vNiceMatrix} 
    %     % \Block[draw,fill=red!15,rounded-corners]{1-1}{}
    %     a_{11} & a_{12} & a_{13} \\ 
    %     a_{21} & 
    %     \Block[draw,fill=red!15,rounded-corners]{2-2}{}
    %     a_{22} & a_{23} & \\ 
    %     a_{31} & a_{32} & a_{33} 
    %   \end{vNiceMatrix}
    %   % \text{M_{11}}
    %   = \tcboxmath[colback=red!15, colframe=red!50!black, arc=1pt, boxrule=0.2pt] { |M_{11}| = a_{22}a_{33} - a_{23}a_{32}} 
    % \end{align*}
    % cofactor는 $|C_{11}|=(-1)^{1+1} |M_{11}|$
  \end{itemize}
\end{frame}




\begin{frame}{행렬식: nxn 행렬 $\Rightarrow$ 라플라스 전개}
  \begin{itemize}
    \item minor $M_{12}$ 은 \\
    \begin{align*}
      \begin{vNiceMatrix} 
        a_{11} & 
        % \Block[draw,fill=blue!15,rounded-corners]{1-1}{}
        a_{12} & a_{13} \\ 
        \Block[draw,fill=blue!15,rounded-corners]{2-1}{}
        a_{21} & 
        a_{22} & 
        \Block[draw,fill=blue!15,rounded-corners]{2-1}{}
        a_{23} \\ 
        a_{31} & a_{32} & a_{33} 
        \CodeAfter
          \tikz \draw [red] (1.5 -|1.2) -- (1.5 -|3.8) ; % horizontal (x,y)
          \tikz \draw [red] (1.2 -|2.5) -- (3.8 -|2.5) ; % vertical
      \end{vNiceMatrix}
      % \text{M_{11}}
      = \tcboxmath[colback=blue!15, colframe=blue!50!black, arc=1pt, boxrule=0.2pt]{|M_{21}| = a_{21}a_{33} - a_{23}a_{31}}
    \end{align*}
    cofactor는 $|C_{12}|=(-1)^{1+2} |M_{12}| = - |M_{12}|$
  \end{itemize}
  따라서, 
  \begin{align*}
    |A|=& \tcboxmath[colback=red!15, colframe=red!50!black, arc=1pt, boxrule=0.2pt]
      {a_{11} \cdot (a_{22}a_{33} - a_{23}a_{32})} 
        - \tcboxmath[colback=blue!15, colframe=blue!50!black, arc=1pt, boxrule=0.2pt]{a_{12} \cdot (a_{21}a_{33} - a_{23}a_{31})} \\
        & + \tcboxmath[colback=green!15, colframe=green!50!black, arc=1pt, boxrule=0.2pt]{a_{13} \cdot (a_{21}a_{32} - a_{22}a_{31})} \\
    =& a_{11} |C_{11}| + a_{12} |C_{12}| + a_{13} |C_{13}| 
    = a_{11} |M_{11}| - a_{12} |M_{12}| + a_{13} |M_{13}| \\
  \end{align*}
\end{frame}





\begin{frame}{행렬식: 4x4 행렬의 전개 (Laplace 전개)}
  \begin{align*}
  &
  \begin{vNiceMatrix}
    \Block[draw,fill=red!15,rounded-corners]{1-1}{}
    a_{11} & a_{12} & a_{13} & a_{14} \\
    a_{21} & \Block[draw,fill=red!15,rounded-corners]{3-3}{}
    a_{22} & a_{23} & a_{24} \\
    a_{31} & a_{32} & a_{33} & a_{34} \\
    a_{41} & a_{42} & a_{43} & a_{44}
    \CodeAfter
      \tikz \draw [red] (1.5 -|1.2) -- (1.5 -|4.8) ; % horizontal (x,y)
      \tikz \draw [red] (1.2 -|1.5) -- (4.8 -|1.5) ; % vertical
  \end{vNiceMatrix}
  -
  \begin{vNiceMatrix}
    a_{11} & 
    \Block[draw,fill=blue!15,rounded-corners]{1-1}{} 
    a_{12} & a_{13} & a_{14} \\
    \Block[draw,fill=blue!15,rounded-corners]{3-1}{}
    a_{21} & a_{22} & 
    \Block[draw,fill=blue!15,rounded-corners]{3-2}{}
    a_{23} & a_{24} \\
    a_{31} & a_{32} & a_{33} & a_{34} \\
    a_{41} & a_{42} & a_{43} & a_{44}
    \CodeAfter
    \tikz \draw [red] (1.5 -|1.2) -- (1.5 -|4.8) ; % horizontal (x,y)
    \tikz \draw [red] (1.2 -|2.5) -- (4.8 -|2.5) ; % vertical
  \end{vNiceMatrix}
  +
  \begin{vNiceMatrix}
    a_{11} & a_{12} & 
    \Block[draw,fill=green!15,rounded-corners]{1-1}{} 
    a_{13} & a_{14} \\
    \Block[draw,fill=green!15,rounded-corners]{3-2}{}
    a_{21} & a_{22} & a_{23} & 
    \Block[draw,fill=green!15,rounded-corners]{3-1}{}
    a_{24} \\
    a_{31} & a_{32} & a_{33} & a_{34} \\
    a_{41} & a_{42} & a_{43} & a_{44}
    \CodeAfter
    \tikz \draw [red] (1.5 -|1.2) -- (1.5 -|4.8) ; % horizontal (x,y)
    \tikz \draw [red] (1.2 -|3.5) -- (4.8 -|3.5) ; % vertical
  \end{vNiceMatrix}
  -
  \begin{vNiceMatrix}
    a_{11} & a_{12} & a_{13} & 
    \Block[draw,fill=orange!15,rounded-corners]{1-1}{} 
    a_{14} \\
    \Block[draw,fill=orange!15,rounded-corners]{3-3}{}
    a_{21} & a_{22} & a_{23} & a_{24} \\
    a_{31} & a_{32} & a_{33} & a_{34} \\
    a_{41} & a_{42} & a_{43} & a_{44}
    \CodeAfter
    \tikz \draw [red] (1.5 -|1.2) -- (1.5 -|4.8) ; % horizontal (x,y)
    \tikz \draw [red] (1.2 -|4.5) -- (4.8 -|4.5) ; % vertical
  \end{vNiceMatrix}
  \\[0.8em]
  =\, &
  \tcboxmath[colback=red!15, colframe=red!50!black, arc=1pt, boxrule=0.2pt]
  {a_{11} \cdot |C_{11}|}
  +
  \tcboxmath[colback=blue!15, colframe=blue!50!black, arc=1pt, boxrule=0.2pt]
  {a_{12} \cdot |C_{12}|}
  +
  \tcboxmath[colback=green!15, colframe=green!50!black, arc=1pt, boxrule=0.2pt]
  {a_{13} \cdot |C_{13}|}
  +
  \tcboxmath[colback=orange!15, colframe=orange!50!black, arc=1pt, boxrule=0.2pt]
  {a_{14} \cdot |C_{14}|}\\
  =\, &
  \tcboxmath[colback=red!15, colframe=red!50!black, arc=1pt, boxrule=0.2pt]
  {a_{11} \cdot (-1)^{1+1}|M_{11}|}
  +
  \tcboxmath[colback=blue!15, colframe=blue!50!black, arc=1pt, boxrule=0.2pt]
  {a_{12} \cdot (-1)^{1+2}|M_{12}|}
  +
  \tcboxmath[colback=green!15, colframe=green!50!black, arc=1pt, boxrule=0.2pt]
  {a_{13} \cdot (-1)^{1+3}|M_{13}|}
  \\
  &+ 
  \tcboxmath[colback=orange!15, colframe=orange!50!black, arc=1pt, boxrule=0.2pt]
  {a_{14} \cdot (-1)^{1+5}|M_{14}|}
  \end{align*}
\end{frame}
  

\begin{frame}{행렬식: nxn 행렬, 4x4 연습}

\end{frame}


\begin{frame}{행렬식: nxn 행렬, 4x4 연습}
  \begin{columns}
    \begin{column}{0.5\textwidth}
      \begin{align*}
        & 
        \begin{bNiceMatrix}
          1 &  2 & 0 &  9 \\
          2 &  3 & 4 &  6 \\
          1 &  6 & 0 & -1 \\
          0 & -5 & 0 &  8 
        \end{bNiceMatrix} 
        \\[1em]
        % 
        &
        \begin{bNiceMatrix}
          2 &  7 & 0 & 1 \\
          5 &  6 & 4 & 8 \\
          0 &  0 & 9 & 0 \\
          1 & 3 & 1 & 4 
        \end{bNiceMatrix}
      \end{align*}
    \end{column}
    \begin{column}{0.5\textwidth}
      \begin{align*}
        & 
        \begin{bNiceMatrix}
          1 &  9 &  2 & 0  \\
          2 &  6 &  3 & 4  \\
          1 & -1 &  6 & 0  \\
          0 &  8 & -5 & 0  
        \end{bNiceMatrix} 
        \\[1em]
        % 
        &
        \begin{bNiceMatrix}
          0 &  0 & 9 & 0 \\
          2 &  7 & 0 & 1 \\
          5 &  6 & 4 & 8 \\
          1 & 3 & 1 & 4 
        \end{bNiceMatrix}
      \end{align*}
    \end{column}
  \end{columns}
\end{frame}


\section{전치행렬}
\begin{frame}{전치행렬 ($A'$, Transposed)}
  \begin{align*}
    A_{m \times n} &=  \begin{bNiceMatrix}
      \CodeBefore
      \tikz \draw [fill=red!15] 
            (1-|2) |- 
            (2-|3) |- 
            (3-|4) |- 
            (4-|5) |- 
            (5-|5) |- 
            cycle ;
      \tikz \draw[fill=blue!15]
            (1-|1) |- 
            (2-|2) |- 
            (3-|3) |- 
            (4-|4) |- 
            (5-|1) |- 
            cycle ;      
      \Body
            a_{11} & a_{12} & \cdots & a_{1n} \\
            a_{21} & a_{22} & \cdots & a_{2n} \\
            \vdots & \vdots & \ddots & \vdots \\
            a_{m1} & a_{m2} & \cdots & a_{mn}
          \end{bNiceMatrix}
  \end{align*}

  \begin{align*}
    {A'}_{m \times n} &=
    \begin{bNiceMatrix}
      \CodeBefore
      \tikz \draw [fill=blue!15] 
          (1-|2) |- 
          (2-|3) |- 
          (3-|4) |- 
          (4-|5) |- 
          (5-|5) |- 
          cycle ;
        \tikz \draw[fill=red!15]
          (1-|1) |- 
          (2-|2) |- 
          (3-|3) |- 
          (4-|4) |- 
          (5-|1) |- 
          cycle ;
      \Body
        a_{11} & a_{21} & \cdots & a_{n1} \\
        a_{12} & a_{22} & \cdots & a_{n2} \\
        \vdots & \vdots & \ddots & \vdots \\
        a_{1m} & a_{2m} & \cdots & a_{nm}
    \end{bNiceMatrix}
    \end{align*}
    
\end{frame}


\section{역행렬}

\begin{frame}{역행렬}
  \begin{block}{역행렬 ($A^{-1}$)}
    행렬 $A$의 역행렬은 $A^{-1}$로 표기하며, $AA^{-1} = A^{-1}A = I$를 만족하는 행렬. \\
    \begin{align*}
      & A^{-1} = \frac{1}{|A|} \text{adj}A = \frac{1}{|A|} C', \\
      & A = \begin{bNiceMatrix}
        \CodeBefore
        \tikz \draw[fill=blue!15]
        (1-|1) |- 
        (2-|2) |- 
        (3-|3) |- 
        (4-|4) |- 
        (5-|1) |- 
        cycle ;      
        \Body
        a_{11} & a_{12} & \cdots & a_{1n} \\
        a_{21} & a_{22} & \cdots & a_{2n} \\
        \vdots & \vdots & \ddots & \vdots \\
        a_{m1} & a_{m2} & \cdots & a_{mn}
      \end{bNiceMatrix}, \quad
      {adj} A = C' = \begin{bNiceMatrix}
        \CodeBefore
        \tikz \draw [fill=blue!15] 
            (1-|2) |- 
            (2-|3) |- 
            (3-|4) |- 
            (4-|5) |- 
            (5-|5) |- 
            cycle ;
        \Body
        |C_{11}| & |C_{21}| & \cdots & |C_{n1}| \\
        |C_{12}| & |C_{22}| & \cdots & |C_{n2}| \\
        \vdots & \vdots & \ddots & \vdots \\
        |C_{1m}| & |C_{2m}| & \cdots & |C_{nm}|
      \end{bNiceMatrix}
    \end{align*}
  \end{block}
  
\end{frame}


\begin{frame}{역행렬: 2x2}
  \begin{align*}
    & A = \begin{bmatrix} a & b \\ c & d \end{bmatrix} , \quad
    A^{-1} = \frac{1}{|A|} \text{adj} A 
    \\
    & \text{adj} A = \begin{bNiceMatrix} |C_{11}|  & |C_{21}| \\ 
                          |C_{12}| & |C_{22}| 
      \end{bNiceMatrix}
      = {\begin{bNiceMatrix} |C_{11}|  & |C_{12}| \\ 
        |C_{21}| & |C_{22}| 
\end{bNiceMatrix}}'
       \\
    & |C_{11}| = (-1)^{1+1} \begin{vNiceMatrix}
      a & b \\ c & d
      \CodeAfter
      \tikz \draw [red] (1.5 -|1.2) -- (1.5 -|2.8) ;
      \tikz \draw [red] (1.2 -|1.5) -- (2.8 -|1.5) ;
    \end{vNiceMatrix} = 1\cdot|d| = d , \quad
    |C_{12}| = (-1)^{1+2} \begin{vNiceMatrix}
      a & b \\ c & d
      \CodeAfter
      \tikz \draw [red] (1.5 -|1.2) -- (1.5 -|2.8) ;
      \tikz \draw [red] (1.2 -|2.5) -- (2.8 -|2.5) ;
    \end{vNiceMatrix} = (-1)\cdot|c| = -c \\
    & |C_{21}| = (-1)^{2+1} \begin{vNiceMatrix}
      a & b \\ c & d
      \CodeAfter
      \tikz \draw [red] (2.5 -|1.2) -- (2.5 -|2.8) ;
      \tikz \draw [red] (1.2 -|1.5) -- (2.8 -|1.5) ;
    \end{vNiceMatrix} = (-1)\cdot|b| = -b , \quad
    |C_{22}| = (-1)^{2+2} \begin{vNiceMatrix}
      a & b \\ c & d
      \CodeAfter
      \tikz \draw [red] (2.5 -|1.2) -- (2.5 -|2.8) ;
      \tikz \draw [red] (1.2 -|2.5) -- (2.8 -|2.5) ;
    \end{vNiceMatrix} = 1\cdot|a| = a \\
    & \therefore A^{-1}= \frac{1}{ad-bc}\begin{bmatrix}
      d & -b \\ -c & a
    \end{bmatrix}
  \end{align*}
\end{frame}


\begin{frame}{역행렬: 3x3}
  \begin{align*}
      & A = 
        \begin{bNiceMatrix}
          a & b & c \\
          d & e & f \\
          g & h & i
        \end{bNiceMatrix} , \quad
        A^{-1} = \frac{1}{\det(A)} \text{adj}(A) 
       = \frac{1}{\det(A)}
        \begin{bNiceMatrix}
          C_{11} & C_{21} & C_{31} \\
          C_{12} & C_{22} & C_{32} \\
          C_{13} & C_{23} & C_{33}
        \end{bNiceMatrix} \\
      % 
      & |C_{11}| = (-1)^{1+1} \begin{vNiceMatrix}
        a & b & c \\ d & e & f \\ g & h & i
            \CodeAfter
              \tikz \draw [red] (1.5 -|1.2) -- (1.5 -|3.8) ; %(vertical)
              \tikz \draw [red] (1.2 -|1.5) -- (3.8 -|1.5) ; %(horizontal)
          \end{vNiceMatrix} = 1\cdot(ei-fh) , \quad  \\
      & |C_{12}| = (-1)^{1+2} \begin{vNiceMatrix}
        a & b & c \\ d & e & f \\ g & h & i
            \CodeAfter
              \tikz \draw [red] (1.5 -|1.2) -- (1.5 -|3.8) ;
              \tikz \draw [red] (1.2 -|2.5) -- (3.8 -|2.5) ;
          \end{vNiceMatrix} = (-1)\cdot(di-fg) , \quad  \\
      & |C_{13}| = (-1)^{1+3} \begin{vNiceMatrix}
        a & b & c \\ d & e & f \\ g & h & i
            \CodeAfter
              \tikz \draw [red] (1.5 -|1.2) -- (1.5 -|3.8) ;
              \tikz \draw [red] (1.2 -|3.5) -- (3.8 -|3.5) ;
          \end{vNiceMatrix} = 1\cdot(dh-eg) , \quad 
      % 
  \end{align*}
\end{frame}

\begin{frame}{역행렬: 3x3}
  \begin{align*}
      & |C_{21}| = (-1)^{2+1} \begin{vNiceMatrix}
        a & b & c \\ d & e & f \\ g & h & i
            \CodeAfter
              \tikz \draw [red] (2.5 -|1.2) -- (2.5 -|3.8) ; %(vertical)
              \tikz \draw [red] (1.2 -|1.5) -- (3.8 -|1.5) ; %(horizontal)
          \end{vNiceMatrix} = (-1)\cdot(bi-ch) , \quad  \\
      & |C_{22}| = (-1)^{2+2} \begin{vNiceMatrix}
        a & b & c \\ d & e & f \\ g & h & i
            \CodeAfter
              \tikz \draw [red] (2.5 -|1.2) -- (2.5 -|3.8) ;
              \tikz \draw [red] (1.2 -|2.5) -- (3.8 -|2.5) ;
          \end{vNiceMatrix} = 1\cdot(ai-cg) , \quad  \\
      & |C_{23}| = (-1)^{2+3} \begin{vNiceMatrix}
        a & b & c \\ d & e & f \\ g & h & i
            \CodeAfter
              \tikz \draw [red] (2.5 -|1.2) -- (2.5 -|3.8) ;
              \tikz \draw [red] (1.2 -|3.5) -- (3.8 -|3.5) ;
          \end{vNiceMatrix} = (-1)\cdot(ah-bg) , \quad 
      % 
  \end{align*}
\end{frame}

\begin{frame}{역행렬: 3x3}
  \begin{align*}
    & |C_{31}| = (-1)^{3+1} \begin{vNiceMatrix}
      a & b & c \\ d & e & f \\ g & h & i
          \CodeAfter
            \tikz \draw [red] (3.5 -|1.2) -- (3.5 -|3.8) ; %(vertical)
            \tikz \draw [red] (1.2 -|1.5) -- (3.8 -|1.5) ; %(horizontal)
        \end{vNiceMatrix} = 1\cdot(bf-ce) , \quad  \\
    & |C_{32}| = (-1)^{3+2} \begin{vNiceMatrix}
      a & b & c \\ d & e & f \\ g & h & i
          \CodeAfter
            \tikz \draw [red] (3.5 -|1.2) -- (3.5 -|3.8) ;
            \tikz \draw [red] (1.2 -|2.5) -- (3.8 -|2.5) ;
        \end{vNiceMatrix} = (-1)\cdot(af-cd) , \quad  \\
    & |C_{33}| = (-1)^{3+3} \begin{vNiceMatrix}
      a & b & c \\ d & e & f \\ g & h & i
          \CodeAfter
            \tikz \draw [red] (3.5 -|1.2) -- (3.5 -|3.8) ;
            \tikz \draw [red] (1.2 -|3.5) -- (3.8 -|3.5) ;
        \end{vNiceMatrix} = 1\cdot(ae-bd) , \quad 
      % 
  \end{align*}
\end{frame}

\begin{frame}{역행렬: 3x3 연습}

\end{frame}


\begin{frame}{역행렬: 3x3 연습}
  \begin{columns}
    \begin{column}{0.5\textwidth}
      \begin{align*}
        A &=
        \begin{bNiceMatrix}
          4 & 1 & -1 \\
          0 & 3 &  2 \\
          3 & 0 &  7 
        \end{bNiceMatrix} , \quad |A| = 99
        \\[1em]
        % 
        C &=
        \begin{bNiceMatrix}
          21 &  6 & -9 \\
          -7 & 31 &  3 \\
           5 & -8 & 12 
        \end{bNiceMatrix}
        \\[1em]
        % 
        adj(A) = C' &=
        \begin{bNiceMatrix}
          21 & -7 &  5 \\
           6 & 31 & -8 \\
          -9 &  3 & 12 
        \end{bNiceMatrix}
      \end{align*}
    \end{column}
    \begin{column}{0.5\textwidth}
      \begin{align*}
        A^{-1} & = \frac{1}{99} 
        \begin{bNiceMatrix}
          21 & -7 &  5 \\
           6 & 31 & -8 \\
          -9 &  3 & 12
        \end{bNiceMatrix} 
        % \\[1em]
        % 
        % &
        % \begin{bNiceMatrix}
        %   0 &  0 & 9 & 0 \\
        %   2 &  7 & 0 & 1 \\
        %   5 &  6 & 4 & 8 \\
        %   1 & 3 & 1 & 4 
        % \end{bNiceMatrix}
      \end{align*}
    \end{column}
  \end{columns}
\end{frame}



\begin{frame}{역행렬: Cramer's Rule}
  % \emph{매번 역행렬을 구하는 것은 매우 귀찮기에.. '해'를를 바로 찾자..}
  \begin{block}{Cramer's Rule}
    % 연립방정식의 해를 찾는 과정을 행렬로 표현하면:\\
    \[
      \begin{bmatrix}
      x_1^* \\
      \vdots \\
      x_j^* \\
      \vdots \\
      x_n^*
      \end{bmatrix}
      =
      \frac{1}{|A|}
      \text{adj}
      \begin{bmatrix}
        a_{11} & a_{12} & \cdots & a_{1j} & \cdots & a_{1n} \\
        \vdots & \vdots &        & \vdots &        & \vdots \\
        a_{j1} & a_{j2} & \cdots & a_{jj} & \cdots & a_{jn} \\
        \vdots & \vdots &        & \vdots &        & \vdots \\
        a_{n1} & a_{n2} & \cdots & a_{nj} & \cdots & a_{nn}
      \end{bmatrix}
      \begin{bNiceMatrix}
        \CodeBefore
        \tikz \draw [fill=blue!15] 
            (1-|2) |- 
            (6-|1) |- 
            cycle ;
        \Body
      d_1 \\
      \vdots \\
      d_j \\
      \vdots \\
      d_n
      \end{bNiceMatrix}
    \] \\
    이 때, 
    \[
      x_j^* = \frac{|\mathbf{A}_j|}{|\mathbf{A}|}
      = \frac{1}{|\mathbf{A}|}
      \begin{vNiceMatrix}
        \CodeBefore
        \tikz \draw [fill=blue!15] 
            (1-|4) |- 
            (5-|5) |- 
            cycle ;
        \Body
      a_{11} & a_{12} & \cdots & d_1 & \cdots & a_{1n} \\
      a_{21} & a_{22} & \cdots & d_2 & \cdots & a_{2n} \\
      \vdots & \vdots &        & \vdots &        & \vdots \\
      a_{n1} & a_{n2} & \cdots & d_n & \cdots & a_{nn}
      \end{vNiceMatrix}
      \]
  \end{block}
\end{frame}

\begin{frame}{역행렬: Cramer's Rule 2x2}
  \begin{align*}
    & Ax=d , \quad A=\begin{bmatrix}
      a_{11} & a_{12} \\ a_{21} & a_{22}
    \end{bmatrix} , \,
    x = \begin{bmatrix}
      x_1 \\ x_2
    \end{bmatrix} , \,
    d = \begin{bmatrix}
      d_1 \\ d_2
    \end{bmatrix} \\
    & x_1^* = \frac{1}{ad-bc} \cdot \begin{vNiceMatrix}
      \CodeBefore
      \tikz \draw [fill=blue!15] 
          (1-|1) |- 
          (3-|2) |- 
          cycle ;
      \Body
      d_1 & a_{12} \\ d_2 & a_{22}
    \end{vNiceMatrix}
    = \frac{1}{ad-bc} \cdot (d_1 a_{22} - a_{12} d_2) \\
    & x_2^* = \frac{1}{ad-bc} \cdot \begin{vNiceMatrix}
      \CodeBefore
      \tikz \draw [fill=blue!15] 
          (1-|2) |- 
          (3-|3) |- 
          cycle ;
      \Body
      a_{11} & d_{1} \\ a_{21} & d_{2}
    \end{vNiceMatrix}
    = \frac{1}{ad-bc} \cdot (a_{11} d2 - d_1 a_{21})
  \end{align*}
  
\end{frame}



\begin{frame}{역행렬: Cramer's Rule 3x3}
  \begin{align*}
    & Ax=d , \quad A=\begin{bmatrix}
      a_{11} & a_{12}  & a_{13} \\ 
      a_{21} & a_{22}  & a_{23} \\ 
      a_{31} & a_{32}  & a_{33} 
    \end{bmatrix} , \,
    x = \begin{bmatrix}
      x_1 \\ x_2 \\ x_3
    \end{bmatrix} , \,
    d = \begin{bmatrix}
      d_1 \\ d_2 \\ d_3
    \end{bmatrix} \\
    & x_1^* = \frac{1}{|A|} \cdot \begin{vNiceMatrix}
      \CodeBefore
      \tikz \draw [fill=blue!15] 
          (1-|1) |- 
          (4-|2) |- 
          cycle ;
      \Body
      d_{1} & a_{12}  & a_{13} \\ 
      d_{2} & a_{22}  & a_{23} \\ 
      d_{3} & a_{32}  & a_{33} 
    \end{vNiceMatrix} \\
    % & x_2^* = \frac{1}{|A|} \cdot \begin{vNiceMatrix}
    %   \CodeBefore
    %   \tikz \draw [fill=blue!15] 
    %       (1-|2) |- 
    %       (5-|3) |- 
    %       cycle ;
    %   \Body
    %   a_{11} & d_{1}  & a_{13} \\ 
    %   a_{21} & d_{2}  & a_{23} \\ 
    %   a_{31} & d_{3}  & a_{33} 
    % \end{vNiceMatrix} \\
    % & x_1^* = \frac{1}{|A|} \cdot \begin{vNiceMatrix}
    %   \CodeBefore
    %   \tikz \draw [fill=blue!15] 
    %       (1-|1) |- 
    %       (4-|2) |- 
    %       cycle ;
    %   \Body
    %   a_{11} & a_{12}  & a_{13} \\ 
    %   a_{21} & a_{22}  & a_{23} \\ 
    %   a_{31} & a_{32}  & a_{33} 
    % \end{vNiceMatrix} \\
    % = \frac{1}{|A|} \cdot (d_1 a_{22} - a_{12} d_2) \\
  %   & x_2^* = \frac{1}{ad-bc} \cdot \begin{vNiceMatrix}
  %     \CodeBefore
  %     \tikz \draw [fill=blue!15] 
  %         (1-|2) |- 
  %         (3-|3) |- 
  %         cycle ;
  %     \Body
  %     a_{11} & d_{1} \\ a_{21} & d_{2}
  %   \end{vNiceMatrix}
  %   = \frac{1}{ad-bc} \cdot (a_{11} d2 - d_1 a_{21})
  \end{align*}
  
\end{frame}

\begin{frame}{역행렬: Cramer’s Rule 연습}
  
\end{frame}


\end{document}